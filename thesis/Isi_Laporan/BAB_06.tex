Pada bagian ini, mahasiswa menjelaskan luaran (\textit{output}) apa yang dijanjikan di akhir periode capstone. Bagian ini harus disinkronkan dengan lembar luaran pada awal dokumen ini. Ada berbagai jenis luaran yang harus dihasilkan minimal satu oleh mahasiswa seperti disebutkan pada lembar luaran.

\textcolor{black}{Sebagaimana telah disinggung di \ref{chap:Dasar_Teori_Pendukung}, mahasiswa juga dituntut untuk mengevaluasi solusi-solusi yang mungkin berdasarkan proses dan standar keteknikan. Untuk mencapai poin tersebut, di Bab ini, mahasiswa harus menentukan kebutuhan, spesifikasi, atau standar dari solusi yang diusulkan di bagian \ref{sec:Spesifikasi_Luaran}. Selain itu, \textbf{mahasiswa harus menentukan kriteria sukses/keberhasilan dari solusi yang dipilih}. Kriteria sukses biasanya didefinisikan jika semua spesifikasi luaran sudah terpenuhi} 

\section{Luaran yang Dijanjikan}
\label{sec:Luaran_yang_Dijanjikan}

    Pada bagian ini, mahasiswa menjelaskan luaran (\textit{output}) apa yang dijanjikan di akhir periode \textit{capstone}. Bagian ini harus disinkronkan dengan lembar luaran pada awal dokumen ini. Ada berbagai jenis luaran yang harus dihasilkan minimal satu oleh mahasiswa seperti disebutkan pada lembar luaran seperti yang dapat dituliskan pada Tabel \ref{tab:Ch06_Contoh_Luaran}.
    
    \begin{longtable}{|L{7.5cm}|L{7.5cm}|}
        \caption{Contoh Luaran} 
        \label{tab:Ch06_Contoh_Luaran}
        \vspace{-0.75em}\\        
        \hline
        \multicolumn{1}{|c|}{\textbf{Jenis Luaran}}                                                                     & \multicolumn{1}{c|}{\textbf{Contoh}}                                                                                \\ \hline
        
        \textit{Hardware} Digital (fisik)                                                                                        & Alat pendeteksi asap berbasis mikroprosesor                                                                         \\ \hline
        
        \textit{Hardware} Analog (fisik)                                                                                         & Alat penghasil frekuensi 1 THz menggunakan rangkaian RLC                                                            \\ \hline
        
        \textit{Firmware}/SW di Mikroprosesor/\textit{Development Board}                                                                  & Alat untuk memonitor asap berbasis komputer                                                                         \\ \hline
        
        \textit{Software} di PC                                                                                                  & Software untuk mendeteksi asap berbasis Python                                                                      \\ \hline
        
        Sistem Informasi                                                                                                & Sistem informasi \textit{capstone} mahasiswa, Sistem informasi manajemen kebakaran                                           \\ \hline
        Simulasi Lengkap                                                                                                & Simulasi dan mitigasi black-out di sistem transmisi 500~kV      \\ \hline
        
        Prototipe/miniatur \textit{hardware}/\textit{software}/sistem                                                                     & Prototipe lampu lalu lintas cerdas perempatan MM-UGM                                                                \\ \hline
        
        Teorema/Teori Baru                                                                                              & Teori kestabilan sistem tenaga terbarukan dengan beban negatif resistif                                             \\ \hline
        
        Kebaruan/\textit{novelty} yang lain/perbaikan metode (apabila berupa \textit{capstone} penelitian)                                & Perbaikan faktor daya dengan metode Alpha, Pencarian jarak terpendek pada sensor asap                               \\ \hline
        
        Dokumen (Kebijakan, SOP, Lingkungan, Ekonomi, dll)                                                              & SOP teknis perencanaan sistem proteksi, Dokumentasi \textit{black-out} dan analisisnya, Dokumen usulan investasi energi, dll \\ \hline
        
        \begin{tabular}[C{7.5cm}]{@{}L{7.5cm}@{}}Lain-lain (sebutkan)\\ Contoh : Simulasi parsial (dengan \textit{software} jadi)\end{tabular} & Contoh : Simulasi trafo dengan Fluent                                                                            \\ \hline
    \end{longtable}
    
\section{Spesifikasi Luaran}
\label{sec:Spesifikasi_Luaran}

    Pada bagian ini, target performa yang akan dicapai berdasarkan fungsionalitas sebagai syarat keberhasilan kinerja produk harus dicantumkan (misal : modul kontroller dapat mengurangi \textit{overshoot} hingga dibawah 1\% atau akurasi produk di bawah 0.01\%). Yang perlu ditekankan disini mahasiswa harus menggunakan standar-standar keteknikan (besaran dan satuan, \textit{policy}/aturan, dan lain-lain) yang disepakati oleh organisasi keteknikan misalnya IEEE, PUIL, ACM dan lain-lain. Contoh spesifikasi luaran yang baik adalah seperti pada Tabel \ref{tab:Ch06_Contoh_Spesifikasi_Luaran}.
    
    \begin{longtable}{|c|c|c|c|c|}
        \caption{Contoh Spesifikasi Luaran} 
        \label{tab:Ch06_Contoh_Spesifikasi_Luaran}
        \vspace{-0.75em}\\
        \hline
        \textbf{No} & \textbf{Spesifikasi}  & \textbf{Satuan} & \textbf{Standar} & \textbf{Keterangan} \\ \hline
        1           & Tegangan Masukan      & Volt (V)        & 95 sampai 220 V  & Lihat Penjelasan A  \\ \hline
        2           & Tegangan Keluaran     & Volt (V)        & 20 V $\pm$ 0,2\% & Lihat Penjelasan B  \\ \hline
        3           & Interferensi Magnetis & Watt (W)        & Maksimal 0,1 W   & Lihat Penjelasan C  \\ \hline
        4           & SNR                   & Decibel (dB)    & Minimum 80 dB    & Lihat Penjelasan D  \\ \hline
        5           & dan seterusnya        &                 &                  &                     \\ \hline
    \end{longtable}
    
  