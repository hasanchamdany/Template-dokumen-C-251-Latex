
Intisari sangat berguna untuk memberikan gambaran singkat tentang keseluruhan isi dokumen, sehingga pembaca dapat memahami secara cepat. Intisari tersusun dari maksimal 3 paragraf dan tidak boleh melebihi 400 kata (1 halaman). Oleh karena itu, mahasiswa harus jeli dalam memilih mana yang harus masuk intisari dan mana yang tidak. Salah satu \underline{good practice} adalah intisari ditulis terakhir kali setelah dokumen C-251 selesai tersusun. Dokumen C-251 dan C-501 boleh ditulis baik dalam Bahasa Indonesia maupun dalam Bahasa Inggris.

\textcolor{black}{Capaian pembelajaran yang diinginkan pada proses penyusunan dokumen C-251 ini adalah
\begin{itemize}
    \item mengidentifikasi suatu permasalahan umum sebagai \textit{complex engineering problem} yang memiliki solusi terbuka (\textit{open ended solutions}),
    \item mengevaluasi solusi-solusi yang mungkin berdasarkan proses dan standar keteknikan, 
    \item  mendesain dan mengalokasikan sumber daya (manusia, fasilitas, keuangan, dan lain-lain) untuk mendukung solusi yang dipilih, serta
    \item mendesain solusi keteknikan untuk permasalahan yang dipilih.
\end{itemize}
}

Pada dokumen C-251 mahasiswa dituntut untuk mampu melakukan analisis Pustaka, memodelkan permasalahan secara matematis/formal, menentukan luaran/produk/metode/simulasi/solusi apa yang harus dibuat untuk menyelesaikan solusi, menentukan spesifikasi luaran/produk/solusi yang dijanjikan tersebut dan melakukan desain umum/pra-desain. Ukuran keberhasilan dari luaran yang dijanjikan pada \textit{Capstone} 2 di semester genap akan menggunakan standar spesifikasi ini. 

Studi pustaka haruslah berhubungan erat dengan topik yang dibahas, sehingga dapat membantu perancangan atas permasalahan yang sudah didefiniskan. Analisis pustaka dibuat setajam mungkin dan seringkas mungkin dengan membahas sedikitnya 3 pustaka kunci dan membandingkan keuntungan dan kerugiannya. Mahasiswa selanjutnya sangat disarankan untuk memformulasikan pemodelan sistemnya baik secara matematis, teknis atau pemodelan lain yang bisa diterima. Setelah itu tim mahasiswa diminta memberikan dasar teori singkat sehingga pembaca dapat memahami analisis pustaka yang dibuat. Kelompok \textit{capstone} diharuskan memilih metode kandidat yang dipilih beserta konsep/teorinya dan memverifikasinya dengan simulasi pendahuluan untuk menentukan apakah desain yang diusulkan benar-benar layak untuk diimplementasikan di dalam C-501. 
